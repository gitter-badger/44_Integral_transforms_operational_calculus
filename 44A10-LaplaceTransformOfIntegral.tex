\documentclass[12pt]{article}
\usepackage{pmmeta}
\pmcanonicalname{LaplaceTransformOfIntegral}
\pmcreated{2014-03-17 10:43:31}
\pmmodified{2014-03-17 10:43:31}
\pmowner{pahio}{2872}
\pmmodifier{pahio}{2872}
\pmtitle{Laplace transform of integral}
\pmrecord{8}{41081}
\pmprivacy{1}
\pmauthor{pahio}{2872}
\pmtype{Derivation}
\pmcomment{trigger rebuild}
\pmclassification{msc}{44A10}

% this is the default PlanetMath preamble.  as your knowledge
% of TeX increases, you will probably want to edit this, but
% it should be fine as is for beginners.

% almost certainly you want these
\usepackage{amssymb}
\usepackage{amsmath}
\usepackage{amsfonts}

% used for TeXing text within eps files
%\usepackage{psfrag}
% need this for including graphics (\includegraphics)
%\usepackage{graphicx}
% for neatly defining theorems and propositions
 \usepackage{amsthm}
% making logically defined graphics
%%%\usepackage{xypic}

% there are many more packages, add them here as you need them

% define commands here

\theoremstyle{definition}
\newtheorem*{thmplain}{Theorem}

\begin{document}
On can show that if a real function \,$t \mapsto f(t)$\, is 
\PMlinkname{Laplace-transformable}{LaplaceTransform}, as well 
is $\displaystyle\int_0^tf(\tau)\,d\tau$.\, The latter is also 
continuous for\, $t > 0$\, and by the 
\PMlinkname{Newton--Leibniz formula}{FundamentalTheoremOfCalculus}, 
has the derivative equal $f(t)$.\, Hence we may apply the 
formula for Laplace transform of derivative, obtaining
$$F(s) \;=\; \mathcal{L}\{f(t)\} \;=\; s\,\mathcal{L} \left\{\int_0^t\!f(\tau)\,d\tau\right\}-\int_0^0\!f(t)\,dt 
\;=\; s\,\mathcal{L} \left\{\int_0^t\!f(\tau)\,d\tau\right\},$$
i.e.
\begin{align}
\mathcal{L} \left\{\int_0^t\!f(\tau)\,d\tau\right\} \;=\; \frac{F(s)}{s}.
\end{align}\\

\textbf{Application.}\, We start from the easily derivable rule
$$\frac{1}{s} \;\curvearrowright\; 1,$$
where the curved \PMlinkescapetext{arrow points} from the Laplace-transformed function to the original function.\, The formula (1) thus yields successively
$$\frac{1}{s^2} \;\curvearrowright\; \int_0^t\!1\,d\tau \;=\; t,$$
$$\frac{1}{s^3} \;\curvearrowright\; \int_0^t\!\tau\,d\tau \;=\; \frac{t^2}{2!},$$
$$\frac{1}{s^4} \;\curvearrowright\; \int_0^t\!\frac{\tau^2}{2!}\,d\tau \;=\; \frac{t^3}{3!},$$
etc.\, Generally, one has
\begin{align}
\frac{1}{s^n} \;\curvearrowright\; \frac{t^{n-1}}{(n\!-\!1)!} \quad \forall\, n \in \mathbb{Z}_+.
\end{align}
%%%%%
%%%%%
\end{document}
