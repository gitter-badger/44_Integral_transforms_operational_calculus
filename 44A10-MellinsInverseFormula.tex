\documentclass[12pt]{article}
\usepackage{pmmeta}
\pmcanonicalname{MellinsInverseFormula}
\pmcreated{2013-03-22 14:23:02}
\pmmodified{2013-03-22 14:23:02}
\pmowner{pahio}{2872}
\pmmodifier{pahio}{2872}
\pmtitle{Mellin's inverse formula}
\pmrecord{13}{35877}
\pmprivacy{1}
\pmauthor{pahio}{2872}
\pmtype{Result}
\pmcomment{trigger rebuild}
\pmclassification{msc}{44A10}
\pmsynonym{inverse Laplace transformation}{MellinsInverseFormula}
\pmsynonym{Bromwich integral}{MellinsInverseFormula}
\pmsynonym{Fourier-Mellin integral}{MellinsInverseFormula}
\pmrelated{InverseLaplaceTransformOfDerivatives}
\pmrelated{HjalmarMellin}
\pmrelated{TelegraphEquation}

\endmetadata

% this is the default PlanetMath preamble.  as your knowledge
% of TeX increases, you will probably want to edit this, but
% it should be fine as is for beginners.

% almost certainly you want these
\usepackage{amssymb}
\usepackage{amsmath}
\usepackage{amsfonts}

% used for TeXing text within eps files
%\usepackage{psfrag}
% need this for including graphics (\includegraphics)
%\usepackage{graphicx}
% for neatly defining theorems and propositions
%\usepackage{amsthm}
% making logically defined graphics
%%%\usepackage{xypic}

% there are many more packages, add them here as you need them

% define commands here
\begin{document}
It may be proven, that if a function $F(s)$ has the {\em inverse Laplace transform} $f(t)$, i.e. a piecewise continuous and exponentially \PMlinkescapetext{restricted} real function $f$ satisfying the condition
   $$\mathcal{L}\{f(t)\} = F(s),$$
then $f(t)$ is uniquely determined when not regarded as different such functions which differ from each other only in a point set having Lebesgue measure zero.  

The inverse Laplace transform is directly given by {\em Mellin's inverse formula}
  $$f(t)= \frac{1}{2\pi i}\int_{\gamma-i\infty}^{\gamma+i\infty}e^{st}F(s)\,ds,$$
by the Finn R. H. Mellin (1854---1933).\, Here it must be integrated along a straight line parallel to the imaginary axis and intersecting the real axis in the point $\gamma$ which must be chosen so that it is greater than the real parts of all singularities of $F(s)$.

In practice, computing the complex integral can be done by using the Cauchy residue theorem.
%%%%%
%%%%%
\end{document}
