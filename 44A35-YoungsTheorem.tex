\documentclass[12pt]{article}
\usepackage{pmmeta}
\pmcanonicalname{YoungsTheorem}
\pmcreated{2013-03-22 18:17:44}
\pmmodified{2013-03-22 18:17:44}
\pmowner{Ziosilvio}{18733}
\pmmodifier{Ziosilvio}{18733}
\pmtitle{Young's theorem}
\pmrecord{15}{40912}
\pmprivacy{1}
\pmauthor{Ziosilvio}{18733}
\pmtype{Theorem}
\pmcomment{trigger rebuild}
\pmclassification{msc}{44A35}

\endmetadata

% this is the default PlanetMath preamble.  as your knowledge
% of TeX increases, you will probably want to edit this, but
% it should be fine as is for beginners.

% almost certainly you want these
\usepackage{amssymb}
\usepackage{amsmath}
\usepackage{amsfonts}

% used for TeXing text within eps files
%\usepackage{psfrag}
% need this for including graphics (\includegraphics)
%\usepackage{graphicx}
% for neatly defining theorems and propositions
%\usepackage{amsthm}
% making logically defined graphics
%%%\usepackage{xypic}

% there are many more packages, add them here as you need them

% define commands here

\begin{document}
%% Last modified: August 10, 2011
%% MSC: 44A35
\newcommand{\ie}{\textit{i.e.}}
\newcommand{\Rset}{\ensuremath{\mathbb{R}}}

\newtheorem{theorem*}{Theorem}

Let $f,g:\Rset^n\to\Rset$.
Recall that the \emph{convolution} of $f$ and $g$ at $x$ is
\begin{displaymath}
(f\ast g)(x)=\int_{\Rset^n}f(x-y)g(y)dy
\end{displaymath}
provided the integral is defined.

The following result is due to William Henry Young.
\begin{theorem*}
Let $p,q,r\in[1,\infty]$ satisfy
\begin{equation} \label{eq:young-coeff}
\frac{1}{p}+\frac{1}{q}-\frac{1}{r}=1
\end{equation}
with the convention $1/\infty=0$.
Let $f\in L^p(\Rset^n)$, $g\in L^q(\Rset^n)$.
Then:
\begin{enumerate}
\item \label{it:L1}
The function $y\mapsto f(x-y)g(y)$
belongs to $L^1(\Rset^n)$ for almost all $x$.
\item \label{it:Lr}
The function $x\mapsto(f\ast g)(x)$
belongs to $L^r(\Rset^n)$.
\item \label{it:cpq}
There exists a constant $c=c_{p,q}\leq 1$,
depending on $p$ and $q$ but not on $f$ or $g$, such that
%%\begin{equation} \label{eq:young-conv}
\begin{displaymath}
\|{f\ast g}\|_r
\leq c\cdot\|f\|_p\cdot\|g\|_q
\end{displaymath}
%%\end{equation}
\end{enumerate}
\end{theorem*}
Observe the analogy with the similar result
with convolution replaced by ordinary (pointwise) product,
where the requirement is $1/p+1/q=1/r$---\ie,
$1/p+1/q-1/r=0$---instead of (\ref{eq:young-coeff}).
The cases
\begin{enumerate}
\item \label{item:p-q-infty}
$1/p+1/q=1$, $r=\infty$
\item \label{item:1-q-q}
$p=1$, $q \in [1,\infty)$, $r=q$
\end{enumerate}
are the most widely known;
for these we provide a proof, supposing $c_{p,q}=1$.
We shall use the following facts:
\begin{itemize}
\item
If $x\mapsto f(x),x\mapsto g(x)$ are measurable,
then $(x,y)\mapsto f(x-y)g(y)$ is measurable.
\item
For any $x$, if $f\in L^p$,
then $y\mapsto f(x-y)$ belongs to $L^p$ as well,
and its $L^p$-norm is the same as $f$'s.
\item
For any $y$, if $f\in L^p$,
then $x\mapsto f(x-y)$ belongs to $L^p$ as well,
and its $L^p$-norm is the same as $f$'s.
\end{itemize}

\textit{Proof of case~\ref{item:p-q-infty}}.

Suppose $f\in L^p(\Rset^n)$, $g\in L^q(\Rset^n)$ with $1/p+1/q=1$.
Then
\begin{displaymath}
\left|\int f(x-y)g(y)dy\right|
\leq\int|f(x-y)g(y)|dy
\leq\|f\|_p\|g\|_q\;.
\end{displaymath}
This holds for all $x\in\Rset^n$,
therefore
\begin{math}
\|f\ast g\|_\infty\leq\|f\|_p\|g\|_q
\end{math}
as well.

\textit{Proof of case~\ref{item:1-q-q}}.

First, suppose $q=1$.
We may suppose $f$ and $g$ are Borel measurable:
if they are not, we replace them with Borel measurable functions
$\tilde{f}$ and $\tilde{g}$
which are equal to $f$ and $g$, respectively,
outside of a set of Lebesgue measure zero;
apply the theorem to $\tilde{f}$, $\tilde{g}$, and $\tilde{f}\ast\tilde{g}$;
and deduce the theorem for $f$, $g$, and $f\ast g$.
By Tonelli's theorem,
\begin{displaymath}
\int\left(\int|f(x-y)g(y)|dy\right)dx
=\int\left(\int|f(x-y)|dx\right)|g(y)|dy
=\|f\|_1\|g\|_1\,,
\end{displaymath}
thus the function $(x,y)\mapsto f(x-y)g(y)$
belongs to $L^1(\Rset^n\times\Rset^n)$.
By Fubini's theorem,
the function $y\mapsto f(x-y)g(y)$
belongs to $L^1(\Rset^n)$ for almost all $x$,
and $x\mapsto(f\ast g)(x)$ belongs to $L^1(\Rset^n)$;
plus,
\begin{displaymath}
\|f\ast g\|_1
\leq\int\int|f(x-y)g(y)|dydx
=\|f\|_1\|g\|_1\;.
\end{displaymath}
Suppose now $q>1$;
choose $q'$ so that $1/q + 1/q' = 1$.
By the argument above,
\begin{math}
y \mapsto |f(x-y)| \cdot |g(y)|^q
\end{math}
belongs to $L^1$ for almost all $x$:
for those $x$, put
\begin{math}
u(y) = |f(x-y)|^{1/q'},
v(y) = |f(x-y)|^{1/q}|g(y)|.
\end{math}
Then $u \in L^{q'}$ and $v \in L^q$ with $1/q' + 1/q = 1$,
so $uv \in L^1$ and
\begin{math}
\|uv\|_1 \leq \|u\|_{q'}\|v\|_q :
\end{math}
but $uv = |f(x-y)g(y)|$, so point~\ref{it:L1} of the theorem is proved.
By H\"older's inequality,
\begin{displaymath}
\left| \int f(x-y)g(y)dy \right|
\leq
\int |f(x-y)g(y)| dy
\leq
\|f\|_1^{1/q'} \left( \int |f(x-y)| \cdot |g(y)|^q dy \right)^{1/q}
\;:
\end{displaymath}
but we know that $|f| \ast |g|^q \in L^1$,
so $f \ast g \in L^q$ and point~\ref{it:Lr} is also proved.
Finally,
\begin{displaymath}
\|f \ast g\|_q^q
\leq
\|f\|_1^{q/q'} \| |f| \ast |g|^q \|_1
\leq
\|f\|_1^{q/q'} \| f \|_1 \| g \|_q^q
=
\|f\|_1^{1+q/q'} \|g\|_q^q
\;:
\end{displaymath}
but $1/q + 1/q' = 1$ means $q+q' = qq'$ and thus $1+q/q' = q$,
so that point~\ref{it:cpq} is also proved.

\begin{thebibliography}{99}

\bibitem{gilardi}
G. Gilardi.
\textit{Analisi tre.}
McGraw-Hill 1994.

\bibitem{rudin}
W. Rudin.
\textit{Real and complex analysis.}
McGraw-Hill 1987.

\bibitem{young1912}
W. H. Young.
On the multiplication of successions of Fourier constants.
\textit{Proc. Roy. Soc. Lond. Series A} \textbf{87} (1912) 331--339.

\end{thebibliography}

%%%%%
%%%%%
\end{document}
