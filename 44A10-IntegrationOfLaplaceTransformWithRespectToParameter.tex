\documentclass[12pt]{article}
\usepackage{pmmeta}
\pmcanonicalname{IntegrationOfLaplaceTransformWithRespectToParameter}
\pmcreated{2013-03-22 18:44:47}
\pmmodified{2013-03-22 18:44:47}
\pmowner{pahio}{2872}
\pmmodifier{pahio}{2872}
\pmtitle{integration of Laplace transform with respect to parameter}
\pmrecord{8}{41521}
\pmprivacy{1}
\pmauthor{pahio}{2872}
\pmtype{Theorem}
\pmcomment{trigger rebuild}
\pmclassification{msc}{44A10}
\pmrelated{TableOfLaplaceTransforms}
\pmrelated{TermwiseDifferentiation}
\pmrelated{MethodsOfEvaluatingImproperIntegrals}
\pmrelated{UsingConvolutionToFindLaplaceTransform}
\pmrelated{RelativeOfCosineIntegral}
\pmrelated{RelativeOfExponentialIntegral}

% this is the default PlanetMath preamble.  as your knowledge
% of TeX increases, you will probably want to edit this, but
% it should be fine as is for beginners.

% almost certainly you want these
\usepackage{amssymb}
\usepackage{amsmath}
\usepackage{amsfonts}

% used for TeXing text within eps files
%\usepackage{psfrag}
% need this for including graphics (\includegraphics)
%\usepackage{graphicx}
% for neatly defining theorems and propositions
 \usepackage{amsthm}
% making logically defined graphics
%%%\usepackage{xypic}

% there are many more packages, add them here as you need them

% define commands here

\theoremstyle{definition}
\newtheorem*{thmplain}{Theorem}

\begin{document}
We use the curved \PMlinkescapetext{arrows to point} from the Laplace-transformed functions to the corresponding initial functions.\\

If\, 
$$f(t,\,x) \;\curvearrowleft\; F(s,\,x),$$
then one can integrate both functions with respect to the parametre $x$ between the same \PMlinkescapetext{limits} which may be also infinite provided that the integrals converge:
\begin{align}
\int_a^b\!f(t,\,x)\,dx \;\curvearrowleft\; \int_a^b\!F(s,\,x)\,dx
\end{align}
(1) may be written as 
\begin{align}
\mathcal{L}\{\int_a^b\!f(t,\,x)\,dx\} \;=\; \int_a^b\!\mathcal{L}\{f(t,\,x)\}\,dx.
\end{align}



{\em Proof.}\, Using the definition of the Laplace transform, we can write
$$\int_a^b\!f(t,\,x)\,dx \;\curvearrowleft\; \int_0^\infty\left(e^{-st}\int_a^b\!f(s,\,x)\,dx\right)dt
\;=\; \int_0^\infty\left(\int_a^b\!e^{-st}f(s,\,x)\,dx\right)dt.$$
We change the \PMlinkescapetext{order} of integration in the last double integral and use again the definition, obtaining
$$\int_a^b\!f(t,\,x)\,dx \;\curvearrowleft\; 
\int_a^b\left(\int_0^\infty\!e^{-st}f(s,\,x)\,dt\right)dx \;=\; \int_a^b\!F(s,\,t)\,dx,$$
Q.E.D.


%%%%%
%%%%%
\end{document}
