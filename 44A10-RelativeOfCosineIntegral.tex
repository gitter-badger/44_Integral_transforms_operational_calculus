\documentclass[12pt]{article}
\usepackage{pmmeta}
\pmcanonicalname{RelativeOfCosineIntegral}
\pmcreated{2013-03-22 18:44:35}
\pmmodified{2013-03-22 18:44:35}
\pmowner{pahio}{2872}
\pmmodifier{pahio}{2872}
\pmtitle{relative of cosine integral}
\pmrecord{8}{41516}
\pmprivacy{1}
\pmauthor{pahio}{2872}
\pmtype{Example}
\pmcomment{trigger rebuild}
\pmclassification{msc}{44A10}
%\pmkeywords{cosine integral}
\pmrelated{SubstitutionNotation}
\pmrelated{EulersConstant}
\pmrelated{RelativeOfExponentialIntegral}
\pmrelated{IntegrationOfLaplaceTransformWithRespectToParameter}

\endmetadata

% this is the default PlanetMath preamble.  as your knowledge
% of TeX increases, you will probably want to edit this, but
% it should be fine as is for beginners.

% almost certainly you want these
\usepackage{amssymb}
\usepackage{amsmath}
\usepackage{amsfonts}

% used for TeXing text within eps files
%\usepackage{psfrag}
% need this for including graphics (\includegraphics)
%\usepackage{graphicx}
% for neatly defining theorems and propositions
%\usepackage{amsthm}
% making logically defined graphics
%%%\usepackage{xypic}

% there are many more packages, add them here as you need them

% define commands here
\newcommand{\sijoitus}[2]%
{\operatornamewithlimits{\Big/}_{\!\!\!#1}^{\,#2}}
\begin{document}
For determining of the value of the improper integral
$$I(a) \;:=\; \int_0^\infty\frac{\cos{ax^2}-\cos{ax}}{x}\,dx \qquad (a > 0),$$
related to the cosine integral, we think it as a function of the parametre $a$ which we denote by $t$.\, Then we can take the Laplace transform (see the integration with respect to a parametre in the table of Laplace transforms):
$$\mathcal{L}\{I(t)\} \;=\; \mathcal{L}\{\int_0^\infty(\cos{tx^2}-\cos{tx})\frac{dx}{x}\} \;=\; 
\int_0^\infty\left(\frac{s}{s^2\!+\!x^4}-\frac{s}{s^2\!+\!x^2}\right)\frac{dx}{x}$$
Splitting the fractional expressions to \PMlinkid{partial fractions}{5812} and integrating give
\begin{align*}
\mathcal{L}\{I(t)\} & \;=\; 
\frac{1}{s}\int_0^\infty\left(\frac{1}{x}-\frac{x^3}{s^2\!+\!x^4}-\frac{1}{x}+\frac{x}{s^2\!+\!x^2}\right)dx\\
 & \;=\; \frac{1}{s}\!\sijoitus{x=0}{\quad\infty}\left[\frac{1}{2}\ln(s^2\!+\!x^2)-\frac{1}{4}\ln(s^2\!+\!x^4)\right] \\
 & \;=\; \frac{1}{4}\!\sijoitus{x=0}{\quad\infty}\ln\frac{(s^2\!+\!x^2)^2}{s^2\!+\!x^4} 
   \;\,=\;\, -\frac{\ln{s}}{2s}.
\end{align*}
As seen in the \PMlinkid{table of Laplace transforms}{10588}, the gotten expression is the Laplace transform of\, 
$\displaystyle\frac{\gamma+\ln{t}}{2} \,=\, I(t)$\; (N.B.\, $\displaystyle\mathcal{L}\{1\} = \frac{1}{s}$), and thus we have the result
$$I(a) \;=\; \frac{\gamma+\ln{a}}{2}.$$
%%%%%
%%%%%
\end{document}
