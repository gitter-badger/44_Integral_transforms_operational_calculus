\documentclass[12pt]{article}
\usepackage{pmmeta}
\pmcanonicalname{LaplaceTransformOfPeriodicFunctions}
\pmcreated{2013-03-22 18:58:24}
\pmmodified{2013-03-22 18:58:24}
\pmowner{pahio}{2872}
\pmmodifier{pahio}{2872}
\pmtitle{Laplace transform of periodic functions}
\pmrecord{5}{41835}
\pmprivacy{1}
\pmauthor{pahio}{2872}
\pmtype{Derivation}
\pmcomment{trigger rebuild}
\pmclassification{msc}{44A10}
\pmrelated{RectificationOfAntiperiodicFunction}
\pmrelated{TableOfLaplaceTransforms}

% this is the default PlanetMath preamble.  as your knowledge
% of TeX increases, you will probably want to edit this, but
% it should be fine as is for beginners.

% almost certainly you want these
\usepackage{amssymb}
\usepackage{amsmath}
\usepackage{amsfonts}

% used for TeXing text within eps files
%\usepackage{psfrag}
% need this for including graphics (\includegraphics)
%\usepackage{graphicx}
% for neatly defining theorems and propositions
 \usepackage{amsthm}
% making logically defined graphics
%%%\usepackage{xypic}

% there are many more packages, add them here as you need them

% define commands here

\theoremstyle{definition}
\newtheorem*{thmplain}{Theorem}

\begin{document}
Let $f(t)$ be periodic with the positive \PMlinkname{period}{PeriodicFunctions} $p$.\, Denote by $H(t)$ the Heaviside step function.\, If now 
$$g(t) \;:=\; f(t)H(t)\!-\!f(t\!-\!p)H(t\!-\!p),$$
then it follows
\begin{align}
g(t) \;=\; 
\begin{cases}
f(t) \quad \mbox{for}\;\;0 < t < p,\\
0 \qquad\, \mbox{otherwise.}
\end{cases}
\end{align}
By the \PMlinkname{parent entry}{DelayTheorem}, the Laplace transform of $g$ is
$$G(s) \;=\; F(s)\!-\!e^{-ps}F(s),$$
whence
$$F(s) \;=\; \frac{G(s)}{1\!-\!e^{-ps}} \;=\; \frac{1}{1\!-\!e^{-ps}}\int_0^\infty\!e^{-st}g(t)\,dt
\;=\; \frac{1}{1\!-\!e^{-ps}}\int_0^p\!e^{-st}f(t)\,dt.$$\\


Thus we have the rule
\begin{align}
\mathcal{L}\{f(t)\} \;=\; \frac{1}{1\!-\!e^{-ps}}\int_0^p\!e^{-st}f(t)\,dt \qquad (\mbox{period } p).
\end{align}


On the contrary, if $f(t)$ is antiperiodic with positive antiperiod $p$, then the function
$$g(t) \;:=\; f(t)H(t)\!+\!f(t\!-\!p)H(t\!-\!p)$$
also has the property (1).\, Analogically with the preceding procedure, one may derive the rule
\begin{align}
\mathcal{L}\{f(t)\} \;=\; \frac{1}{1\!+\!e^{-ps}}\int_0^p\!e^{-st}f(t)\,dt \qquad (\mbox{antiperiod } p).
\end{align}



%%%%%
%%%%%
\end{document}
