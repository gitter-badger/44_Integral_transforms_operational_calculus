\documentclass[12pt]{article}
\usepackage{pmmeta}
\pmcanonicalname{InverseLaplaceTransformOfDerivatives}
\pmcreated{2013-03-22 16:46:27}
\pmmodified{2013-03-22 16:46:27}
\pmowner{pahio}{2872}
\pmmodifier{pahio}{2872}
\pmtitle{inverse Laplace transform of derivatives}
\pmrecord{11}{39003}
\pmprivacy{1}
\pmauthor{pahio}{2872}
\pmtype{Derivation}
\pmcomment{trigger rebuild}
\pmclassification{msc}{44A10}
\pmsynonym{differentiation of Laplace transform}{InverseLaplaceTransformOfDerivatives}
\pmrelated{MellinsInverseFormula}
\pmrelated{SeparationOfVariables}
\pmrelated{KalleVaisala}
\pmrelated{TableOfLaplaceTransforms}

% this is the default PlanetMath preamble.  as your knowledge
% of TeX increases, you will probably want to edit this, but
% it should be fine as is for beginners.

% almost certainly you want these
\usepackage{amssymb}
\usepackage{amsmath}
\usepackage{amsfonts}

% used for TeXing text within eps files
%\usepackage{psfrag}
% need this for including graphics (\includegraphics)
%\usepackage{graphicx}
% for neatly defining theorems and propositions
 \usepackage{amsthm}
% making logically defined graphics
%%%\usepackage{xypic}

% there are many more packages, add them here as you need them

% define commands here

\theoremstyle{definition}
\newtheorem*{thmplain}{Theorem}

\begin{document}
It may be shown that the Laplace transform \,$F(s) = 
\int _0^\infty e^{-st}f(t)\,dt$\, is always differentiable and that its derivative can be formed by \PMlinkname{differentiating under the integral sign}{DifferentiationUnderIntegralSign}, i.e. one has
 $$F'(s) = \int_0^\infty\frac{\partial(e^{-st}f(t))}{\partial s}\,dt = 
\int_0^\infty e^{-st}(-t)f(t)\,dt.$$
This gives the rule
\begin{align}
    \mathcal{L}^{-1}\{F'(s)\} = -tf(t).
\end{align}
Applying (1) to $F'(s)$ instead of $F(s)$ gives
     $$\mathcal{L}^{-1}\{F''(s)\} = t^2f(t).$$
Continuing this way we can obtain the general rule
\begin{align}
    \mathcal{L}^{-1}\{F^{(n)}(s)\} = (-1)^nt^nf(t),
\end{align}
or equivalently
\begin{align}
 \mathcal{L}\{t^nf(t)\} = (-1)^n\cdot\frac{d^n\mathcal{L}\{f(t)\}}{ds^n},
\end{align}
for any\, $n = 1,\,2,\,3,\,\ldots$ (and of course for\, $n = 0$).\\

\textbf{Example.}\, Let's find the Laplace transform of the first kind and 0th \PMlinkescapetext{order} Bessel function 
$$ J_{0}(t) := 
\sum_{m=0}^\infty \frac{(-1)^m}{(m!)^2}\left(\frac{t}{2}\right)^{2m},
$$
which is the solution $y(t)$ of the Bessel's equation 
\begin{align}
    ty''(t)+y'(t)+ty(t) = 0
\end{align}
satisfying the initial condition\, $y(0) = 1$.\, The equation implies that\, $y'(0) = 0$.

By (3), the Laplace transform of the differential equation (4) is
$$-\frac{d\mathcal{L}\{y''(t)\}}{ds}+\mathcal{L}\{y'(t)\}
-\frac{d\mathcal{L}\{y(t)\}}{ds} = 0.$$
Using here twice the rule 5 in the \PMlinkname{parent}{LaplaceTransform} entry gives us
$$-\frac{d(s^2Y(s)-s)}{ds}+sY(s)-1-\frac{dY(s)}{ds} = 0,$$
which is simplified to
$$(s^2+1)\frac{dY}{ds}+sY = 0,$$
i.e. to
$$\frac{dY}{Y} = -\frac{s\,ds}{s^2+1}.$$
Integrating this gives  
$$\ln Y = -\frac{1}{2}\ln(s^2+1)+\ln C = \ln\frac{C}{\sqrt{s^2+1}},$$
i.e.
   $$Y(s) = \frac{C}{\sqrt{s^2+1}}.$$
The initial condition enables to justify that the integration constant $C$ must be 1.\, Thus we have the result
$$\mathcal{L}\{J_0(t)\} = \frac{1}{\sqrt{s^2+1}}.$$ 


\begin{thebibliography}{9}
\bibitem{K.V.}{\sc K. V\"ais\"al\"a:} {\em Laplace-muunnos}.\, Handout Nr. 163.\quad Teknillisen korkeakoulun ylioppilaskunta, Otaniemi, Finland (1968).
\end{thebibliography}

%%%%%
%%%%%
\end{document}
