\documentclass[12pt]{article}
\usepackage{pmmeta}
\pmcanonicalname{FeketesSubadditiveLemma}
\pmcreated{2014-03-19 22:15:10}
\pmmodified{2014-03-19 22:15:10}
\pmowner{Filipe}{28191}
\pmmodifier{Filipe}{28191}
\pmtitle{Fekete's subadditive lemma}
\pmrecord{4}{88067}
\pmprivacy{1}
\pmauthor{Filipe}{28191}
\pmtype{Theorem}

% this is the default PlanetMath preamble.  as your knowledge
% of TeX increases, you will probably want to edit this, but
% it should be fine as is for beginners.

% almost certainly you want these
\usepackage{amssymb}
\usepackage{amsmath}
\usepackage{amsfonts}

% need this for including graphics (\includegraphics)
\usepackage{graphicx}
% for neatly defining theorems and propositions
\usepackage{amsthm}

% making logically defined graphics
%\usepackage{xypic}
% used for TeXing text within eps files
%\usepackage{psfrag}

% there are many more packages, add them here as you need them

% define commands here

\begin{document}
Let $(a_n)_n$ be a subadditive sequence in $[-\infty,\infty)$. Then, the following limit exists in $[-\infty,\infty)$ and equals the infimum of the same sequence:
$$\lim_{n} \frac{a_n}{n}= \inf_n \frac{a_n}{n}$$
Although the lemma is usually stated for subadditive sequences, an analogue conclusion is valid for superadditive sequences. In that case, for $(a_n)_n$ a subadditive sequence in $(-\infty,\infty]$, one has:
$$\lim_n \frac{a_n}{n}=\sup_n \frac{a_n}{n}$$
The proof of the superadditive case is obtained by taking the symmetric sequence $(-a_n)_n$ and applying the subadditive version of the theorem.
\end{document}
