\documentclass[12pt]{article}
\usepackage{pmmeta}
\pmcanonicalname{DifferentiationOfLaplaceTransformWithRespectToParameter}
\pmcreated{2014-03-09 12:52:11}
\pmmodified{2014-03-09 12:52:11}
\pmowner{pahio}{2872}
\pmmodifier{pahio}{2872}
\pmtitle{differentiation of Laplace transform with respect to parameter}
\pmrecord{2}{88064}
\pmprivacy{1}
\pmauthor{pahio}{2872}
\pmtype{Theorem}
\pmclassification{msc}{44A10}

% this is the default PlanetMath preamble.  as your knowledge
% of TeX increases, you will probably want to edit this, but
% it should be fine as is for beginners.

% almost certainly you want these
\usepackage{amssymb}
\usepackage{amsmath}
\usepackage{amsfonts}

% need this for including graphics (\includegraphics)
\usepackage{graphicx}
% for neatly defining theorems and propositions
\usepackage{amsthm}

% making logically defined graphics
%\usepackage{xypic}
% used for TeXing text within eps files
%\usepackage{psfrag}

% there are many more packages, add them here as you need them

% define commands here

\begin{document}
We use the curved \PMlinkescapetext{arrows to point} from the Laplace-transformed functions to the corresponding initial functions.\\

If\, 
$$f(t,x) \;\,\curvearrowleft\;\, F(s,x),$$
then one can differentiate both functions with respect to the parametre $x$:
\begin{align}
f'_x(t,x)  \;\,\curvearrowleft\;\, F'_x(s,x)
\end{align}
(1) may be written also as 
\begin{align}
\mathcal{L}\{\frac{\partial}{\partial x}f(t,x)}\}
\;=\; \frac{\partial}{\partial x}\mathcal{L}\{f(t,x)\}.
\end{align}

{\em Proof.}\, We differentiate partially both sides of the defining 
equation
$$F(s,x) \;:= \int_0^\infty e^{-st}f(t,x)\,dt,$$
on the right hand side 
\PMlinkname{under the integration sign}{differentiationundertheintegralsign}, getting
\begin{align}
F'_x(s,x) \;=\; \int_0^\infty e^{-st}f'_x(t,x)\,dx,
\end{align}
which means same as (1).\, Q.E.D.\\

\textbf{Example.}\, If the rule
$$\frac{s}{s^2\!-\!a^2} \;\,\curvearrowright\;\, \cosh{at}$$
is differentiated with respect to $a$, the result is
$$\frac{2as}{(s^2\!-\!a^2)^2}\;\,\curvearrowright
\;\, t\,\sinh{at}.$$\\

\begin{thebibliography}{9}
\bibitem{K.V.}{\sc K. V\"ais\"al\"a:} {\em Laplace-muunnos}.\, Handout Nr. 163.\quad Teknillisen korkeakoulun ylioppilaskunta, Otaniemi, Finland (1968).
\end{thebibliography}


\end{document}
