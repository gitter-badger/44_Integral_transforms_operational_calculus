\documentclass[12pt]{article}
\usepackage{pmmeta}
\pmcanonicalname{LaplaceTransformsOfDerivatives}
\pmcreated{2014-04-06 8:24:33}
\pmmodified{2014-04-06 8:24:33}
\pmowner{pahio}{2872}
\pmmodifier{pahio}{2872}
\pmtitle{Laplace transforms of derivatives}
\pmrecord{6}{88063}
\pmprivacy{1}
\pmauthor{pahio}{2872}
\pmtype{Topic}
\pmclassification{msc}{44A10}

\endmetadata

where% this is the default PlanetMath preamble.  as your knowledge
% of TeX increases, you will probably want to edit this, but
% it should be fine as is for beginners.

% almost certainly you want these
\usepackage{amssymb}
\usepackage{amsmath}
\usepackage{amsfonts}

% need this for including graphics (\includegraphics)
\usepackage{graphicx}
% for neatly defining theorems and propositions
\usepackage{amsthm}

% making logically defined graphics
%\usepackage{xypic}
% used for TeXing text within eps files
%\usepackage{psfrag}

% there are many more packages, add them here as you need them

% define commands here

\begin{document}
As shown in the \PMlinkname{parent entry}{LaplaceTransformOfDerivative},
the Laplace transform of the first derivative of a Laplace-transformable 
function $f(t)$ is got from the formula
\begin{align}
\mathcal{L}\{f'(t)\} \;=\; s\,F(s)-\lim_{t\to0+}\!f(t).
\end{align}
The rule can be applied also to the function $f'(t)$:
$$\mathcal{L}\{f''(t)\} \;=\; s[sF(s)-\lim_{t\to0+}\!f(t)]-\lim_{t\to0+}\!f'(t) 
\;=\; s^2F(s)-sf(0\!^+)-f'(0\!^+)$$
Here the short notation $0\!^+$ has been used for the right limits.

Further, one can use the rule to $f''(t)$, getting
$$\mathcal{L}\{f'''(t)\} \;=\; s[s^2F(s)-sf(0\!^+)-f'(0\!^+)]-f''(0\!^+) 
\;=\; s^3F(s)-s^2f(0\!^+)-sf'(0\!^+)-f''(0\!^+).$$
Continuing similarly, one comes to the general formula
\begin{align}
\mathcal{L}\{f^{(n)}(t)\} 
\;=\;  s^nF(s)-s^{n-1}f(0\!^+)-s^{n-2}f'(0\!^+)-\ldots-f^{(n-1)}(0\!^+).
\end{align}
Use of (2) requires that $f(t)$, $f'(t)$, $f''(t)$, ..., $f^{(n)}(t)$ are 
Laplace-transformable and that $f(t)$, $f'(t)$, $f''(t)$, ..., 
$f^{(n-1)}(t)$ are continuous when\, $t > 0$ (not only 
piecewise continuous).\\

\textbf{Remark.}\, Suppose that $f(t)$ and $f'(t)$ are 
Laplace-transformable and that $f(t)$ is continuous for\, 
$t > 0$\, except the point\, $t = a$\, where the function has a 
finite jump discontinuity.\, Then
$$\mathcal{L}\{f'(t)\} \;=\; 
sF(s)-f(0\!^+)-e^{-as}(\lim_{s\to a+}f(s)-\lim_{s\to a-}f(s)).$$\\



\textbf{Application.}\, Derive the Laplace transform of $\sin{at}$ using the 
derivatives of sine (cf. Laplace transform of cosine and sine).

We have
$$f(t) \;:=\; \sin{at}, \qquad f'(t) \;=\; a\cos{at}, \qquad
f''(t) \;=\; -a^2\sin{at}.$$
Using (2) with\, $n =2$\, we obtain
$$\mathcal{L}\{-a^2\sin{at}\} 
\;=\; s^2\mathcal{L}\{\sin{at}\}-s\sin0-a\cos0,$$
i.e.
$$-a^2\mathcal{L}\{\sin{at}\} \;=\; s^2\mathcal{L}\{\sin{at}\}-a,$$
which implies
$$\mathcal{L}\{\sin{at}\} \;=\; \frac{a}{s^2\!+\!a^2}.$$
\end{document}
