\documentclass[12pt]{article}
\usepackage{pmmeta}
\pmcanonicalname{RelativeOfExponentialIntegral}
\pmcreated{2013-03-22 18:44:20}
\pmmodified{2013-03-22 18:44:20}
\pmowner{pahio}{2872}
\pmmodifier{pahio}{2872}
\pmtitle{relative of exponential integral}
\pmrecord{12}{41511}
\pmprivacy{1}
\pmauthor{pahio}{2872}
\pmtype{Example}
\pmcomment{trigger rebuild}
\pmclassification{msc}{44A10}
\pmclassification{msc}{26A36}
\pmrelated{SubstitutionNotation}
\pmrelated{RelativeOfCosineIntegral}
\pmrelated{IntegrationOfLaplaceTransformWithRespectToParameter}
\pmrelated{IntegrationUnderIntegralSign}

\endmetadata

% this is the default PlanetMath preamble.  as your knowledge
% of TeX increases, you will probably want to edit this, but
% it should be fine as is for beginners.

% almost certainly you want these
\usepackage{amssymb}
\usepackage{amsmath}
\usepackage{amsfonts}

% used for TeXing text within eps files
%\usepackage{psfrag}
% need this for including graphics (\includegraphics)
%\usepackage{graphicx}
% for neatly defining theorems and propositions
%\usepackage{amsthm}
% making logically defined graphics
%%%\usepackage{xypic}

% there are many more packages, add them here as you need them

% define commands here
\newcommand{\sijoitus}[2]%
{\operatornamewithlimits{\Big/}_{\!\!\!#1}^{\,#2}}
\begin{document}
Let $a$ and $b$ be positive numbers.\, We want to calculate the value of the improper integral
\begin{align}
\int_0^\infty\frac{e^{-ax}-e^{-bx}}{x}\,dx
\end{align}
related to the exponential integral.\\

The value may be found e.g. by utilising the derivative of the integral
$$I(y) \;:=\, \int_0^\infty e^{-xy}\!\cdot\!\frac{e^{-ax}-e^{-bx}}{x}\,dx$$
which can be formed by \PMlinkname{differentiating under the integral sign}{DifferentiationUnderIntegralSign}:
\begin{align*}
I'(y) & \;=\; \int_0^\infty e^{-xy}(-x)\frac{e^{-ax}-e^{-bx}}{x}\,dx\\
      & \;=\; \int_0^\infty\left(e^{-(y+b)x}-e^{-(y+a)x}\right)\,dx\\
      & \;=\; \sijoitus{x=0}{\quad\infty}\!\left(\frac{e^{-(y+b)x}}{-(y\!+\!b)}-\frac{e^{-(y+a)x}}{-(y\!+\!a)}\right)\\
      & \;=\; \frac{1}{y\!+\!b}-\frac{1}{y\!+\!a}
\end{align*}
Thus,
$$I(y) \;=\; \ln(y\!+\!b)-\ln(y\!+\!a) \;=\; \ln\frac{y\!+\!b}{y\!+\!a},$$
and the integral (1) has the value\, $\displaystyle I(0) = \ln\frac{b}{a}$.\\

There is another method via Laplace transforms.\, By the table of Laplace transforms, we have
$$\mathcal{L}\{e^{-at}-e^{-bt}\} \;=\; \frac{1}{s\!+\!a}-\frac{1}{s\!+\!b}$$
and therefore
$$\mathcal{L}\{\frac{e^{-at}-e^{-bt}}{t}\} \;=\; \int_s^\infty\left(\frac{1}{u\!+\!a}-\frac{1}{u\!+\!b}\right)\,du 
\;=\; \sijoitus{u=s}{\quad\infty}\ln\frac{u\!+\!a}{u\!+\!b} \;=\; \ln\frac{s\!+\!b}{s\!+\!a},$$
i.e.
$$\int_0^\infty e^{-st}\!\cdot\!\frac{e^{-at}-e^{-bt}}{t}\,dt \;=\; \ln\frac{s\!+\!b}{s\!+\!a}.$$
Letting\, $s \to 0+$,\, this yields the equation
$$\int_0^\infty\frac{e^{-ax}-e^{-bx}}{x}\,dx \;=\; \ln\frac{b}{a}.$$

%%%%%
%%%%%
\end{document}
