\documentclass[12pt]{article}
\usepackage{pmmeta}
\pmcanonicalname{LaplaceTransformOffracftt}
\pmcreated{2014-03-08 15:45:15}
\pmmodified{2014-03-08 15:45:15}
\pmowner{pahio}{2872}
\pmmodifier{pahio}{2872}
\pmtitle{Laplace transform of $\frac{f(t)}{t}$}
\pmrecord{8}{40918}
\pmprivacy{1}
\pmauthor{pahio}{2872}
\pmtype{Derivation}
\pmcomment{trigger rebuild}
\pmclassification{msc}{44A10}
\pmrelated{FundamentalTheoremOfCalculusClassicalVersion}
\pmrelated{SubstitutionNotation}
\pmrelated{CyclometricFunctions}

% this is the default PlanetMath preamble.  as your knowledge
% of TeX increases, you will probably want to edit this, but
% it should be fine as is for beginners.

% almost certainly you want these
\usepackage{amssymb}
\usepackage{amsmath}
\usepackage{amsfonts}

% used for TeXing text within eps files
%\usepackage{psfrag}
% need this for including graphics (\includegraphics)
%\usepackage{graphicx}
% for neatly defining theorems and propositions
%\usepackage{amsthm}
% making logically defined graphics
%%%\usepackage{xypic}

% there are many more packages, add them here as you need them

% define commands here
\DeclareMathOperator{\arccot}{arccot}
\newcommand{\sijoitus}[2]%
{\operatornamewithlimits{\Big/}_{\!\!\!#1}^{\,#2}}
\begin{document}
Suppose that the quotient
$$\frac{f(t)}{t} \,:=\; g(t)$$
is \PMlinkname{Laplace-transformable}{LaplaceTransform}.\, It follows easily that also $f(t)$ is such.\, According to the \PMlinkname{parent entry}{LaplaceTransformOfTnft}, we may write
$$\mathcal{L}^{-1}\left\{G'(s)\right\} \,=\, -t\,g(t) \,=\, -f(t) \,=\, \mathcal{L}^{-1}\left\{-F(s)\right\}.$$
Therefore
$$G'(s) \,=\, -F(s),$$
whence
\begin{align}
G(s) \,=\, -F^{(-1)}(s)+C
\end{align}
where $F^{(-1)}(s)$ means any antiderivative of $F(s)$.\, Since each Laplace transformed function vanishes in the infinity \,$s = \infty$\, and thus\, $G(\infty) = 0$,\, the equation (1) implies
$$C \,=\, F^{(-1)}(\infty)$$
and therefore
$$G(s) \,=\, F^{(-1)}(\infty)\!-\!F^{(-1)}(s) \,=\, \int_s^\infty F(u)\,du.$$
We have obtained the result
\begin{align}
\mathcal{L}\left\{\frac{f(t)}{t}\right\} \,=\, \int_s^\infty F(u)\,du.
\end{align}


\textbf{Application.}\, By the table of Laplace transforms,\, 
$\displaystyle\mathcal{L}\left\{\sin{t}\right\} = \frac{1}{s^2+1}.$\, Accordingly the formula (2) yields
$$\mathcal{L}\left\{\frac{\sin{t}}{t}\right\} = \int_s^{\,\infty}\!\frac{1}{u^2+1}\,du
= \sijoitus{s}{\quad\infty}\!\arctan{u} \,=\, \frac{\pi}{2}\!-\!\arctan{s} \,=\, \arccot{s}.$$
Thus we have
\begin{align}
\mathcal{L}\left\{\frac{\sin{t}}{t}\right\} \,=\, \arccot{s} \,=\, \arctan\frac{1}{s}.
\end{align}
This result is derived in the entry Laplace transform of sine integral in two other ways.




%%%%%
%%%%%
\end{document}
