\documentclass[12pt]{article}
\usepackage{pmmeta}
\pmcanonicalname{LaplaceTransformOfCosineAndSine}
\pmcreated{2013-03-22 18:18:27}
\pmmodified{2013-03-22 18:18:27}
\pmowner{pahio}{2872}
\pmmodifier{pahio}{2872}
\pmtitle{Laplace transform of cosine and sine}
\pmrecord{8}{40929}
\pmprivacy{1}
\pmauthor{pahio}{2872}
\pmtype{Derivation}
\pmcomment{trigger rebuild}
\pmclassification{msc}{44A10}
\pmsynonym{Laplace transform of sine and cosine}{LaplaceTransformOfCosineAndSine}
%\pmkeywords{Laplace transform of cosine}
%\pmkeywords{Laplace transform of sine}

% this is the default PlanetMath preamble.  as your knowledge
% of TeX increases, you will probably want to edit this, but
% it should be fine as is for beginners.

% almost certainly you want these
\usepackage{amssymb}
\usepackage{amsmath}
\usepackage{amsfonts}

% used for TeXing text within eps files
%\usepackage{psfrag}
% need this for including graphics (\includegraphics)
%\usepackage{graphicx}
% for neatly defining theorems and propositions
 \usepackage{amsthm}
% making logically defined graphics
%%%\usepackage{xypic}

% there are many more packages, add them here as you need them

% define commands here

\theoremstyle{definition}
\newtheorem*{thmplain}{Theorem}

\begin{document}
We start from the easily \PMlinkescapetext{derivable} formula
\begin{align}
e^{\alpha t} \;\curvearrowleft\; \frac{1}{s\!-\!\alpha} \qquad (s > \alpha),
\end{align}
where the curved \PMlinkescapetext{arrow points} from the Laplace-transformed function to the original function.\, Replacing $\alpha$ by $-\alpha$ we can write the second formula
\begin{align}
e^{-\alpha t} \;\curvearrowleft\; \frac{1}{s\!+\!\alpha} \qquad (s > -\alpha).
\end{align}
Adding (1) and (2) and dividing by 2 we obtain (remembering the linearity of the Laplace transform)
$$\frac{e^{\alpha t}\!+\!e^{-\alpha t}}{2} \;\curvearrowleft\; 
\frac{1}{2}\!\left(\frac{1}{s\!-\!\alpha}\!+\!\frac{1}{s\!+\!\alpha}\right),$$
i.e.
\begin{align}
\mathcal{L}\{\cosh{\alpha t}\} = \frac{s}{s^2\!-\!\alpha^2}.
\end{align}
Similarly, subtracting (1) and (2) and dividing by 2 give
\begin{align}
\mathcal{L}\{\sinh{\alpha t}\} = \frac{a}{s^2\!-\!\alpha^2}.
\end{align}
The formulae (3) and (4) are valid for\, $s > |\alpha|$.\\


There are the hyperbolic identities
$$\cosh{it} = \cos{t}, \quad \frac{1}{i}\sinh{it} = \sin{t}$$
which enable the transition from hyperbolic to trigonometric functions.\, If we choose\, $\alpha := ia$\, in (3), we may calculate
$$\cos{at} = \cosh{iat} \;\curvearrowleft\; \frac{s}{s^2\!-\!(ia)^2} = \frac{s}{s^2+a^2},$$
the formula (4) analogously gives
$$\sin{at} = \frac{1}{i}\sinh{iat} \;\curvearrowleft\; \frac{1}{i}\!\cdot\!\frac{ia}{s^2\!-\!(ia)^2} = \frac{a}{s^2+a^2}.$$
Accordingly, we have derived the Laplace transforms
\begin{align}
\mathcal{L}\{\cos{at}\} = \frac{s}{s^2\!+\!a^2},
\end{align}
\begin{align}
\mathcal{L}\{\sin{at}\} = \frac{a}{s^2\!+\!a^2},
\end{align}
which are true for\, $s > 0$.

%%%%%
%%%%%
\end{document}
