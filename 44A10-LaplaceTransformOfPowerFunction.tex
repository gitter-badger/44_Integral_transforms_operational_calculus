\documentclass[12pt]{article}
\usepackage{pmmeta}
\pmcanonicalname{LaplaceTransformOfPowerFunction}
\pmcreated{2013-03-22 18:17:42}
\pmmodified{2013-03-22 18:17:42}
\pmowner{pahio}{2872}
\pmmodifier{pahio}{2872}
\pmtitle{Laplace transform of power function}
\pmrecord{6}{40911}
\pmprivacy{1}
\pmauthor{pahio}{2872}
\pmtype{Derivation}
\pmcomment{trigger rebuild}
\pmclassification{msc}{44A10}
\pmrelated{EvaluatingTheGammaFunctionAt12}

% this is the default PlanetMath preamble.  as your knowledge
% of TeX increases, you will probably want to edit this, but
% it should be fine as is for beginners.

% almost certainly you want these
\usepackage{amssymb}
\usepackage{amsmath}
\usepackage{amsfonts}

% used for TeXing text within eps files
%\usepackage{psfrag}
% need this for including graphics (\includegraphics)
%\usepackage{graphicx}
% for neatly defining theorems and propositions
 \usepackage{amsthm}
% making logically defined graphics
%%%\usepackage{xypic}

% there are many more packages, add them here as you need them

% define commands here

\theoremstyle{definition}
\newtheorem*{thmplain}{Theorem}

\begin{document}
\PMlinkescapeword{integral}
In the defining \PMlinkname{integral}{ImproperIntegral}
$$\mathcal{L}\left\{t^r\right\} \;=\; \int_0^\infty\!e^{-st}t^r\,dt$$
of the Laplace transform of the power function \,$t \mapsto t^r$,\, we make the \PMlinkname{substitution}{SubstitutionForIntegration} \,$u := st$:
$$\mathcal{L}\left\{t^r\right\} \;=\; \int_0^\infty\!e^{-u}\left(\frac{u}{s}\right)^r\frac{du}{s} 
\;=\; \frac{1}{s^{n+1}}\!\int_0^\infty\!e^{-u}u^{r+1-1}\,du$$
Here we have assumed that\, $r > -1$\, and $s > 0$.\, According to the definition of the gamma function, the last integral is equal to 
$\Gamma(r\!+\!1)$.\, Thus we obtain
\begin{align}
\mathcal{L}\left\{t^r\right\} \;=\; \frac{\Gamma(r\!+\!1)}{s^{r+1}}.
\end{align}


The special case\, $r = -\frac{1}{2}$\, gives the result
\begin{align}
\mathcal{L}\left\{\frac{1}{\sqrt{t}}\right\} \;=\; \sqrt{\frac{\pi}{s}}.
\end{align}
%%%%%
%%%%%
\end{document}
