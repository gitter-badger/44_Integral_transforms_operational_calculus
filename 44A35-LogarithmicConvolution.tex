\documentclass[12pt]{article}
\usepackage{pmmeta}
\pmcanonicalname{LogarithmicConvolution}
\pmcreated{2013-03-22 14:28:26}
\pmmodified{2013-03-22 14:28:26}
\pmowner{swiftset}{1337}
\pmmodifier{swiftset}{1337}
\pmtitle{logarithmic convolution}
\pmrecord{4}{35995}
\pmprivacy{1}
\pmauthor{swiftset}{1337}
\pmtype{Definition}
\pmcomment{trigger rebuild}
\pmclassification{msc}{44A35}
\pmsynonym{scale convolution}{LogarithmicConvolution}
\pmrelated{Convolution}

% this is the default PlanetMath preamble.  as your knowledge
% of TeX increases, you will probably want to edit this, but
% it should be fine as is for beginners.

% almost certainly you want these
\usepackage{amssymb}
\usepackage{amsmath}
\usepackage{amsfonts}

% used for TeXing text within eps files
%\usepackage{psfrag}
% need this for including graphics (\includegraphics)
%\usepackage{graphicx}
% for neatly defining theorems and propositions
%\usepackage{amsthm}
% making logically defined graphics
%%%\usepackage{xypic}

% there are many more packages, add them here as you need them

% define commands here
\begin{document}
\paragraph{Definition}

The \emph{scale convolution} of two functions $s(t)$ and $r(t)$, also known as their \emph{logarithmic convolution} is defined as the function 
$$ s \ast_l r(t) = r \ast_l s(t) = \int_0^\infty s(\frac{t}{a})r(a) \frac{da}{a} $$
when this quantity exists.

\paragraph{Results}

The logarithmic convolution can be related to the ordinary convolution by changing the variable from $t$ to $v = \log t$:
\begin{eqnarray*}
s \ast_l r(t) & = & \int_0^\infty s(\frac{t}{a})r(a) \frac{da}{a} = 
                \int_{-\infty}^\infty s(\frac{t}{e^u}) r(e^u) du \\
 & = & \int_{-\infty}^\infty s(e^{\log t - u})r(e^u) du
\end{eqnarray*}
Define $f(v) = s(e^v)$ and $g(v) = r(e^v)$ and let $v = \log t$, then
$$ s \ast_l r(v) = f \ast g(v) = g \ast f(v) = r \ast_l s(v). $$
%%%%%
%%%%%
\end{document}
