\documentclass[12pt]{article}
\usepackage{pmmeta}
\pmcanonicalname{MellinTransform}
\pmcreated{2015-02-17 15:10:57}
\pmmodified{2015-02-17 15:10:57}
\pmowner{rspuzio}{6075}
\pmmodifier{pahio}{2872}
\pmtitle{Mellin transform}
\pmrecord{7}{40589}
\pmprivacy{1}
\pmauthor{rspuzio}{2872}
\pmtype{Definition}
\pmcomment{trigger rebuild}
\pmclassification{msc}{44A15}

% this is the default PlanetMath preamble.  as your knowledge
% of TeX increases, you will probably want to edit this, but
% it should be fine as is for beginners.

% almost certainly you want these
\usepackage{amssymb}
\usepackage{amsmath}
\usepackage{amsfonts}

% used for TeXing text within eps files
%\usepackage{psfrag}
% need this for including graphics (\includegraphics)
%\usepackage{graphicx}
% for neatly defining theorems and propositions
%\usepackage{amsthm}
% making logically defined graphics
%%%\usepackage{xypic}

% there are many more packages, add them here as you need them

% define commands here

\begin{document}
The \emph{Mellin transform} is an integral transform defined as follows:
\[
 F(s) = \int_0^\infty f(t) t^{s-1} \, dt
\]
Intuitively, it may be viewed as a continuous analogue of a power series 
--- instead of synthetizing a function by summing multiples of integer
powers, we integrate over all real powers.  This transform is closely
related to the Laplace transform --- if we make a change of variables
$t = e^{-r}$  and define $g$ by $f(e^{-r}) = g(r)$, then the above integral 
becomes
\[
 F(s) = -\int_{-\infty}^{+\infty} g(r) e^{-rs} \, dr ,
\]
which is a bilateral Laplace transform.

(more to come)
%%%%%
%%%%%
\end{document}
