\documentclass[12pt]{article}
\usepackage{pmmeta}
\pmcanonicalname{RectificationOfAntiperiodicFunction}
\pmcreated{2013-03-22 18:58:19}
\pmmodified{2013-03-22 18:58:19}
\pmowner{pahio}{2872}
\pmmodifier{pahio}{2872}
\pmtitle{rectification of antiperiodic function}
\pmrecord{6}{41832}
\pmprivacy{1}
\pmauthor{pahio}{2872}
\pmtype{Definition}
\pmcomment{trigger rebuild}
\pmclassification{msc}{44A10}
\pmclassification{msc}{26A99}
\pmsynonym{rectification}{RectificationOfAntiperiodicFunction}
\pmrelated{MaximalNumber}
\pmrelated{LaplaceTransformOfPeriodicFunctions}
\pmrelated{MinimalAndMaximalNumber}
\pmdefines{half-wave rectification}
\pmdefines{full-wave rectification}

\endmetadata

% this is the default PlanetMath preamble.  as your knowledge
% of TeX increases, you will probably want to edit this, but
% it should be fine as is for beginners.

% almost certainly you want these
\usepackage{amssymb}
\usepackage{amsmath}
\usepackage{amsfonts}

% used for TeXing text within eps files
%\usepackage{psfrag}
% need this for including graphics (\includegraphics)
%\usepackage{graphicx}
% for neatly defining theorems and propositions
 \usepackage{amsthm}
% making logically defined graphics
%%%\usepackage{xypic}

% there are many more packages, add them here as you need them

% define commands here

\theoremstyle{definition}
\newtheorem*{thmplain}{Theorem}

\begin{document}
Let the positive number $p$ be the antiperiod of the real function $f$ and
$$f(t) \geqq 0 \quad \mbox{for} \;\; 0 < t < p.$$
Then the function $f_1$ defined by
\begin{align*}
f_1(t) \;:=\; \max\{f(t),\,0\} \;=\;
\begin{cases}
f(t) \quad \mbox{for} \;\; f(t) > 0,\\
0 \qquad\, \mbox{for} \;\; f(t) \leqq 0
\end{cases}
\end{align*}
is the \emph{half-wave rectification} of $f$ and the function $f_2$ defined by
$$f_2(t) \;:=\; |f(t)|$$
is the \emph{full-wave rectification} of $f$.\, They are \PMlinkname{periodic}{PeriodicFunctions}, the former with \PMlinkname{period}{PeriodicFunctions} $2p$ and the latter with \PMlinkescapetext{period} $p$.\\


The Laplace transforms are
$$\mathcal{L}\{f_1(t)\} \;=\; \frac{1}{1\!-\!e^{-ps}}F(s),$$
$$\mathcal{L}\{f_2(t)\} \;=\; \frac{1\!+\!e^{-ps}}{1\!-\!e^{-ps}}F(s).$$


%%%%%
%%%%%
\end{document}
