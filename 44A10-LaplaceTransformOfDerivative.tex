\documentclass[12pt]{article}
\usepackage{pmmeta}
\pmcanonicalname{LaplaceTransformOfDerivative}
\pmcreated{2013-03-22 18:24:54}
\pmmodified{2013-03-22 18:24:54}
\pmowner{pahio}{2872}
\pmmodifier{pahio}{2872}
\pmtitle{Laplace transform of derivative}
\pmrecord{5}{41065}
\pmprivacy{1}
\pmauthor{pahio}{2872}
\pmtype{Theorem}
\pmcomment{trigger rebuild}
\pmclassification{msc}{44A10}
\pmrelated{SubstitutionNotation}

% this is the default PlanetMath preamble.  as your knowledge
% of TeX increases, you will probably want to edit this, but
% it should be fine as is for beginners.

% almost certainly you want these
\usepackage{amssymb}
\usepackage{amsmath}
\usepackage{amsfonts}

% used for TeXing text within eps files
%\usepackage{psfrag}
% need this for including graphics (\includegraphics)
%\usepackage{graphicx}
% for neatly defining theorems and propositions
%\usepackage{amsthm}
% making logically defined graphics
%%%\usepackage{xypic}

% there are many more packages, add them here as you need them

% define commands here
\newcommand{\sijoitus}[2]%
{\operatornamewithlimits{\Big/}_{\!\!\!#1}^{\,#2}}
\begin{document}
\textbf{Theorem.}\; If the real function \,$t \mapsto f(t)$\, and its derivative are Laplace-transformable and $f$ is continuous for\, $t > 0$,\, then
\begin{align}
\mathcal{L}\{f'(t)\} \;=\; s\,F(s)-\lim_{t\to0+}\!f(t).
\end{align}

{\em Proof.}\; By the definition of Laplace transform and using integration by parts, the left hand side of (1) may be written
$$\int_0^\infty\!e^{-st}f'(t)\,dt 
\;=\;  \sijoitus{t=0}{\quad \infty}\!e^{-st}f(t)+s\!\int_0^\infty\!e^{-st}f(t)\,dt
\;=\; \lim_{t\to\infty}e^{-st}f(t)-\lim_{t\to0}e^{-st}f(t)+s\,F(s).$$
The Laplace-transformability of $f$ implies that $e^{-st}f(t)$ tends to zero as $t$ increases boundlessly.\, Thus the last expression leads to the right hand side of (1).
%%%%%
%%%%%
\end{document}
